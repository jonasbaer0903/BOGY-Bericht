\chapter{Praktikum}

\section{Zeitlicher Ablauf}

Ich begann meine Praktikumstage im Zeitraum von 8:30 bis 9:00 Uhr und beendete diese um ca. 16:00 Uhr. Die Mittagspause fand jeden Tag unterschiedlich nach Bedarf statt und dauerte ca. 30 Minuten. 

\section{Tätigkeiten/Ablauf}

Mein Praktikum begann damit, dass ich im Eingangsbereich von KIS zusammen mit einem Anderen BOGY-Schülerpraktikanten von einer der zwei Ausbildungsbeauftragten von KIS abgeholt wurde und wir in das Gebäude der SIT gingen, wo sie eine Präsentation über Kaufland und die Schwarzgruppe hielt, uns unsere Praktikumsmappen, die einen Praktikumsplan und Vorlagen für Tagesberichte enthielten, und unsere Schlüsselkarten gab.

Danach liefen wir zurück zur KIS und sie brachte uns in unsere Bereiche. Dort angekommen erhielt ich zusammen mit Carsten, der Sngewandte Informatik an der DHBW Mosbach im 4. Semester dual studiert, und Felix, der ebenfalls an der DHBW Mosbach im 4. Semester Angewandte Informatik dual studiert, von meinem \frqq Betreuer\flqq \ Heiko eine Einführung in Java. Darauf folgte eine weitere Präsentation, die sich aber vor allem an die Studenten richtete und ihnen ein wichtiges Element ihres Projekts, einem online Test für Kaufland Mitarbeiter und Führungskräfte, der ihnen zeigen soll was sie \frqq antreibt\flqq \ ihre Leistung zu bringen und ihnen Aufschluss darüber gibt ob dieser oder diese Antriebe zu stark, zu schwach oder im \frqq grünen Bereich\flqq \ sind, erklärte. Nach den zwei Präsentationen erhielt ich meine erste Aufgabe, die Programmierung eines Taschenrechners, der die 4 Grundrechenarten beherrscht, in der Programmiersprache Java, welche ich bis zum Ende des ersten Tages fast vollständig bearbeitet hatte. Am nächsten Tag stellte ich dies fertig und begann mit der Programmierung einer grafischen Oberfläche, wozu ich das Internet zur Hilfe nahm. Nach der Mittagspause wurde ich dann für 45 Minuten von Frau Bechtel besucht. Darauf wurde mir das Filialportal vorgestellt, das Mitarbeiter in den Kaufland Märkten benutzen um z. B. Preisschilder zu erstellen und zu drucken. Den Rest des Tages befasste ich mich mit meinem Taschenrechner, den ich am Mittwoch fertigstellte. Am Mittwoch besuchte ich eine Schulung zum zum Schreiben von wissenschaftlichen Arbeiten geeigneten Programm LaTeX, mit dem ich auch diesen Bericht erstellt habe. Am Nachmittag des gleichen Tages begann ich mit der Verfassung dieses Berichts, was ich am Donnerstag fortsetzte. 

\section{Persönliche Erfahrungen}

Ich habe durch mein Praktikum Erfahrungen gemacht, wie in einem großen IT-Unternehmen wie KIS gearbeitet wird und zum anderen konnte ich selbst mit dualen Studenten sprechen, wie sie ihr Studium bei der DHBW und Kaufland beurteilen. Durch meine Eigenständigen Aufgaben habe ich außerdem Erfahrungen im Umgang mit Java und LaTex gesammelt die mir in meinem Studium höchst wahrscheinlich von großem Nutzen sein werden. Auch habe ich erfahren, dass die Arbeitszeiten und Abläufe, nicht wie von mir erwartet festgelegt sind, sondern sich relativ flexibel gestalten. 

\section{Probleme}

Während meines Praktikums hatte ich weder organisatorische, noch zwischenmenschliche oder persönliche Probleme. 

\section{Arbeitsergebnisse}

Meine Arbeitsergebnisse liegen diesem Bericht in Form von Screenshots von den Taschenrechnerprogrammen, der LaTex Version dieses Berichts und einer LaTex Übung von der gleichnamigen Schulung an. 